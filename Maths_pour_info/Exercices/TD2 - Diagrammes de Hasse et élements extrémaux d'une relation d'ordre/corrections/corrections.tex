\documentclass[a4paper,12pt]{article}


\usepackage[top=2.5cm, bottom=2.5cm, left=3cm, right=3cm]{geometry}
\setlength{\parskip}{10pt}

%% Mandatory stuff
\usepackage[utf8]{inputenc}
\usepackage[T1]{fontenc}

%% Math package
\usepackage{amsmath}
\usepackage{amssymb}
\usepackage{amsfonts}
\usepackage{mathrsfs}
\DeclareMathAlphabet{\mathpzc}{OT1}{pzc}{m}{it}% Define a particular math font

%Advanced tabular managers
\usepackage{array}
\usepackage{multirow}

% Improve itemize environment
\usepackage{enumitem}
\setlist[itemize]{noitemsep, topsep=0pt}% Delete useless space

% Figure managers
\usepackage{graphicx}
\usepackage{float}

%% Color package (for links, source code when using listings,...)
\usepackage[usenames]{color}
	%% Color for hyperlinks
\definecolor{colHyperlinks}{RGB}{0,90,170}%{59,0,159}

%
% Define pseudo-code environment and its layout
%
\usepackage{algorithm}
\usepackage{algpseudocode}

\errorcontextlines\maxdimen

% begin vertical rule patch for algorithmicx (http://tex.stackexchange.com/questions/144840/vertical-loop-block-lines-in-algorithmicx-with-noend-option)
\makeatletter
% start with some helper code
% This is the vertical rule that is inserted
\newcommand*{\algrule}[1][\algorithmicindent]{\makebox[#1][l]{\hspace*{.5em}\vrule height .75\baselineskip depth .25\baselineskip}}%

\newcount\ALG@printindent@tempcnta
\def\ALG@printindent{%
    \ifnum \theALG@nested>0% is there anything to print
        \ifx\ALG@text\ALG@x@notext% is this an end group without any text?
            % do nothing
            \addvspace{-3pt}% FUDGE for cases where no text is shown, to make the rules line up
        \else
            \unskip
            % draw a rule for each indent level
            \ALG@printindent@tempcnta=1
            \loop
                \algrule[\csname ALG@ind@\the\ALG@printindent@tempcnta\endcsname]%
                \advance \ALG@printindent@tempcnta 1
            \ifnum \ALG@printindent@tempcnta<\numexpr\theALG@nested+1\relax% can't do <=, so add one to RHS and use < instead
            \repeat
        \fi
    \fi
    }%
\usepackage{etoolbox}
% the following line injects our new indent handling code in place of the default spacing
\patchcmd{\ALG@doentity}{\noindent\hskip\ALG@tlm}{\ALG@printindent}{}{\errmessage{failed to patch}}
\makeatother
% end vertical rule patch for algorithmicx

% French translation of each keyword
\algrenewcommand\algorithmicend{\textbf{Fin}}
\algrenewcommand\algorithmicdo{\textbf{faire}}
\algrenewcommand\algorithmicwhile{\textbf{Tant que}}
\algrenewcommand\algorithmicfor{\textbf{Pour}}
\algrenewcommand\algorithmicforall{\textbf{Pour chaque}}
\algrenewcommand\algorithmicloop{\textbf{Boucler}}
\algrenewcommand\algorithmicrepeat{\textbf{Répéter}}
\algrenewcommand\algorithmicuntil{\textbf{Jusqu'à}}
\algrenewcommand\algorithmicprocedure{\textbf{Procédure}}
\algrenewcommand\algorithmicfunction{\textbf{Fonction}}
\algrenewcommand\algorithmicif{\textbf{Si}}
\algrenewcommand\algorithmicthen{\textbf{alors}}
\algrenewcommand\algorithmicelse{\textbf{Sinon}}
\algrenewcommand\algorithmicrequire{\textbf{Entrée:}}
\algrenewcommand\algorithmicensure{\textbf{Sortie:}}
\algrenewcommand\algorithmicreturn{\textbf{Retourner}}
\algrenewcommand\textproc{\textsc}
% Define new commands to begin and end an algorithm
\algnewcommand\algorithmicbegin{\textbf{Début}}
\algnewcommand\Begin{\item[\algorithmicbegin]}
\algnewcommand\algorithmicendbegin{\textbf{Fin}}
\algnewcommand\EndBegin{\item[\algorithmicendbegin]}

%% Hyperlinks for references (must be the last package loaded)
\usepackage[pdftex,%
  colorlinks,%
  linkcolor=colHyperlinks,%
  urlcolor=colHyperlinks,%
  citecolor=colHyperlinks,
  plainpages=false]{hyperref}
\begin{document}

\begin{center}
\Large{Licence L2\\Mathématiques pour l'informatique \\Correction d'exercices du TD 2}
\end{center}

\section*{Exercice 13}

On considère $A = \mathbb{N}$ muni d'une relation d'ordre $|$

\begin{minipage}{0.5\textwidth}
	a) Soit $B = \{1, 2, 3\}$

	\begin{itemize}
		\item Ensemble des maximaux : $\{2, 3\}$
		\item $\text{max}(B)$ n'existe pas
		\item Ensemble des minimaux : $\{1\}$
		\item $\text{min}(B) = 1$
		\item Ensemble des majorants : $\{6\mathbb{N}\}$
		\item $\text{sup}(B) = 6$
		\item Ensemble des minorants : $\{1\}$
		\item $\text{inf}(B) = 1$
	\end{itemize}
\end{minipage}
\begin{minipage}{0.5\textwidth}
	b) Soit $B = \{0, 2, 4, 8\}$

	\begin{itemize}
		\item Ensemble des maximaux : $\{0\}$
		\item $\text{max}(B) = 0$
		\item Ensemble des minimaux : $\{2\}$
		\item $\text{min}(B) = 2$
		\item Ensemble des majorants : $\{0\}$
		\item $\text{sup}(B) = 0$
		\item Ensemble des minorants : $\{1, 2\}$
		\item $\text{inf}(B) = 2$
	\end{itemize}
\end{minipage}

\begin{minipage}{0.5\textwidth}
	c) Soit $B = \{4, 6, 8\}$

	\begin{itemize}
		\item Ensemble des maximaux : $\{6, 8\}$
		\item $\text{max}(B)$ n'existe pas.
		\item Ensemble des minimaux : $\{4, 6\}$
		\item $\text{min}(B)$ n'existe pas.
		\item Ensemble des majorants : $\{24\mathbb{N}\}$
		\item $\text{sup}(B) = 24$
		\item Ensemble des minorants : $\{1, 2\}$
		\item $\text{inf}(B) = 2$
	\end{itemize}
\end{minipage}
\begin{minipage}{0.5\textwidth}
	d) Soit $B = \{7\}$

	\begin{itemize}
		\item Ensemble des maximaux : $\{7\}$
		\item $\text{max}(B) = 7$.
		\item Ensemble des minimaux : $\{7\}$
		\item $\text{min}(B) = 7$.
		\item Ensemble des majorants : $\{7\mathbb{N}\}$
		\item $\text{sup}(B) = 7$
		\item Ensemble des minorants : $\{1, 7\}$
		\item $\text{inf}(B) = 7$
	\end{itemize}
\end{minipage}

e) Soit $B = \{3\mathbb{N}\}$

\begin{itemize}
	\item Ensemble des maximaux = Ensemble des majorants = $\{0\}$
	\item $\text{max}(B) = \text{sup}(B) = 0$
	\item Ensemble des minimaux : $\{3\}$
	\item $\text{min}(B) = 3$.
	\item Ensemble des minorants : $\{1, 3\}$
	\item $\text{inf}(B) = 3$
\end{itemize}

\newpage

\section*{Exercice 15}

\begin{algorithm}
	\caption{maximal(n, M, B, i) de type booléen}
	\begin{algorithmic}[1]
\Require 

	$n$ : card$(A)$

	$M$ : matrice booléenne associée à  la relation d'ordre d'indice [0...n-1][0...n-1]

	$B$ : tableau caractéristique de $B\subset A$ d'indice [0...n-1]

	$i$ : élément de $A$ à  tester
\Ensure 

	vrai si $i$ est un élément maximal de $B$

\Begin
	\If{ $B[i]$ }
		\State $j \leftarrow 0$
		\While{$j<i$ et (!$B[j]$ ou !M[i][j])} \Comment Voir ci-après
			\State $j \leftarrow j + 1$
		\EndWhile
		\If{$j = 1$}
			\State $j \leftarrow j + 1$
			\While{$j<n$ et (!$B[j]$ ou !M[i][j])} \Comment Voir ci-après
				\State $j \leftarrow j + 1$
			\EndWhile
			\State \Return $(j=n)$
		\EndIf
	\EndIf
	\State \Return faux

	\Comment Si on inverse les indices de la matrice afin d'avoir M[j][i], on obtient l'algorithme "minimal"
\EndBegin
	\end{algorithmic}
\end{algorithm}

\begin{algorithm}
	\caption{minimum(n, M, B, i) de type booléen}
	\begin{algorithmic}[1]

	\Require 

	$n$ : card$(A)$

	$M$ : matrice booléenne associée à  la relation d'ordre d'indice [0...n-1][0...n-1]

	$B$ : tableau caractéristique de $B\subset A$ d'indice [0...n-1]

	$i$ : élément de $A$ à  tester
\Ensure 

	vrai si $i$ est le minimum de $B$

\Begin
	\If{$B[i]$}
		\State $j \leftarrow 0$
		
		\Comment Si on inverse les indices de la matrice afin d'avoir M[j][i], on obtient l'algorithme "maximum"
		\While{$j<n$ et (!$B[j]$ ou M[i][j])}
			\State $j \leftarrow j + 1$
		\EndWhile
		\State \Return $(j=n)$
	\EndIf
	\State \Return faux
\EndBegin
	\end{algorithmic}
\end{algorithm}

\begin{algorithm}
	\caption{major(n, M, B, i) de type booléen}
	\begin{algorithmic}[1]

	\Require 

	$n$ : card$(A)$

	$M$ : matrice booléenne associée à  la relation d'ordre d'indice [0...n-1][0...n-1]

	$B$ : tableau caractéristique de $B\subset A$ d'indice [0...n-1]

	$i$ : élément de $A$ à  tester
\Ensure 

	vrai si $i$ est un majorant de $B$

\Begin
	\If{$B[i]$}
		\State test $\leftarrow$ vrai
		\State $j \leftarrow 0$
		
		\While{$j<n$ et test}
		
			\Comment En inversant les indices de la matrice afin d'avoir M[i][j], on obtient l'algorithme "minorant"
			\State test $\leftarrow$ B[j] $\leq$ M[j][i]
			\State $j \leftarrow j + 1$
		\EndWhile
	\EndIf
	\State \Return test
\EndBegin
	\end{algorithmic}
\end{algorithm}

\end{document}