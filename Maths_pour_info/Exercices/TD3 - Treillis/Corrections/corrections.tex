\documentclass[a4paper,12pt]{article}


\usepackage[top=2.5cm, bottom=2.5cm, left=3cm, right=3cm]{geometry}
\setlength{\parskip}{10pt}

%% Mandatory stuff
\usepackage[utf8]{inputenc}
\usepackage[T1]{fontenc}

%% Math package
\usepackage{amsmath}
\usepackage{amssymb}
\usepackage{amsfonts}
\usepackage{mathrsfs}
\DeclareMathAlphabet{\mathpzc}{OT1}{pzc}{m}{it}% Define a particular math font

\begin{document}

\begin{center}
\Large{Licence L2\\Mathématiques pour l'informatique \\Correction d'un exercice du TD 3}
\end{center}

\section*{Exercice 21}

1) Soit T un treillis, a et x deux éléments de T. Montrer que :
$$ a \vee (x \wedge a) = a \text{ et } a \wedge (x \vee a) = a \text{ (propriété d'absoprtion)}$$

Montrons que $a \vee (x \wedge a) = a$ :

On sait que $a \leq a \vee (x \wedge a)$ et que $x \wedge a \leq a$. Donc $a \vee (x \wedge a) \leq a$

Ainsi $a \vee (x \wedge a) = \text{max}(a, x \wedge a) = a $

Montrons que $a \wedge (x \vee a) = a$ :

On sait que $a \geq a \wedge (x \vee a)$ et que $x \vee a \geq a$. Donc $a \wedge (x \vee a) \geq a$

Ainsi $a \wedge (x \vee a) = \text{min}(a, x \vee a) = a $

2) $\forall x, y, z \in T$, montrer que :
$$x \leq z \Rightarrow x \vee (y \wedge z) \leq (x \vee y) \wedge z$$
Quand a-t-on l'égalité?

D'une part $x \wedge z \leq z$ et $x \leq z$ donc $x \vee (y \wedge z) \leq z$ (a).

D'autre part, $y \wedge z \leq y$ donc $x \vee (y \wedge z) \leq x \vee z$ (b).

Ainsi, d'après (a) et (b), on en déduit que $x \vee (y \wedge z) \leq (x \vee y) \wedge z$

3) Montrer les règles de distributivité "faibles" :

$x \vee (y \wedge z) \leq (x \vee y) \wedge (x \vee z)$ (a)

$(x \wedge y) \vee (x \wedge z) \leq x \wedge (y \vee z)$ (b)


a) $y \wedge z \leq y$ et $y \wedge z \leq z$ donc $x \vee (y \wedge z) \leq x \vee y$ et $x \vee (y \wedge z) \leq x \vee z$

Donc $x \vee (y \wedge z) \leq (x \vee y) \wedge (x \vee z)$

b) $y \leq y \vee z$ et $z \leq y \vee z$ donc $x \wedge y \leq x \wedge (y \vee z)$ et $x \wedge z \leq x \wedge (y \vee z)$

Donc $(x \wedge y) \vee (x \wedge z) \leq x \wedge (y \vee z)$

4) Montrer que si 
$$\forall x, y, z \in T, x \wedge (y \vee z) = (x \wedge y) \vee (x \wedge z)$$
alors on :
$$\forall x, y, z \in T, x \vee (y \wedge z) = (x \vee y) \wedge (x \vee z)$$

Soit $(x \vee y) \wedge (x \vee z) = \text{ (D'après l'hypothèse) } ((x \vee y) \wedge x) \vee ((x \vee y) \wedge z)$

D'après 1) et également d'après l'hypothèse, on a :

$ = x \vee ((x \wedge z) \vee (y \wedge z))$

Par association :

$ = (x \vee (x \wedge z)) \vee (y \wedge z)$

Encore d'après 1), on a :

$ = x \vee (y \wedge z)$
\end{document}